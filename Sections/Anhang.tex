\clearpage
\onecolumn
\begin{landscape}
\section{Anhang}
\subsection{Rauschen ZF Tabelle}\label{ZF_rauschen}
\textbf{Achtung:} Siehe \script{185} für Vorraussetzungen
	\renewcommand{\arraystretch}{1.5}
	\begin{longtable}{|c|c|c|c|c|c|c|}
		\hline
		& Baseband 
		& DSB-SC    
		& AM Coherent 
		& AM Envelope  
		& PM  
		& FM          \\
		\hline
		Nachrichtensignal  
		& \multicolumn{6}{c|}
		{Zufallsprozess $X(t)$ mit $\left| X(t) \right| \leq 1$
			bzw. $\left| x_{\lambda}(t) \right| \leq 1$ f\"ur alle $\lambda$ des Ergebnisraums $S$} \\
		\hline
		Leistung $S_{X}$ von $X(t)$
		& \multicolumn{6}{c|}
		{$S_{X} = S_{X}(t) = E\left[ X^{2}(t)\right] \leq 1$,
			(weil $\left| X(t) \right| \leq 1$)}\\
		\hline
		Bandbreite von $X(t)$ 
		& \multicolumn{6}{c|}{B} \\
		\hline
		Eingangsnutzsignal $X_{i}(t)$
		& $X(t)$ 
		& $X(t) A_{c}\cos(\omega_{c}t)$
		& \multicolumn{2}{c|}{$A_{c}(1+\mu X(t))\cos(\omega_{c}t)$} 
		& \multicolumn{1}{c|} {$A_{c}\cos(\omega_{c}t + k_{p}X(t))$} 
		& {$A_{c}\cos(\omega_{c}t + k_{f}\int\limits_{-\infty}^{t} X(\tau)\;d\tau)$}  \\
		\hline
		Leistung $S_{i}$ von $X_{i}(t)$ 
		& $S_{X}$
		& $\frac{1}{2}A_{c}^{2} S_{X}$
		& \multicolumn{2}{c|}{$\frac{1}{2}A_{c}^{2} (1 + \mu^{2}S_{X}) $}
		& \multicolumn{1}{c|} {$\frac{1}{2}A_{c}^{2}$}
		& {$\frac{1}{2}A_{c}^{2}$} \\
		\hline
		Bandbreite von $X_{i}(t)$ 
		& $B$
		& $2B$
		& \multicolumn{2}{c|}{$2B$}
		& \multicolumn{1}{c|}{$2(D + 1) B$}
		& {$2(D + 1) B$} \\
		\hline
		Rauschleistung am Eingang
		& $\eta B$
		& $2\eta B$
		& \multicolumn{2}{c|}{$2\eta B$}
		& \multicolumn{1}{c|}{$2(D + 1)\eta B$}
		& {$2(D + 1)\eta B$} \\
		\hline
		SNR am Eingang $\left(\frac{S}{N}\right)_{i}$
		& $\frac{S_{i}}{\eta B}$
		& $\frac{\frac{1}{2}A_{c}^{2} S_{X}}{2\eta B}$
		& \multicolumn{2}{c|}{$\frac{\frac{1}{2}A_{c}^{2} (1 + \mu^{2}S_{X})}{2\eta B}$}
		& \multicolumn{1}{c|}{$\frac{\frac{1}{2}A_{c}^{2}}{2(D + 1)\eta B}$}
		& {$\frac{\frac{1}{2}A_{c}^{2}}{2(D + 1)\eta B}$} \\
		\hline
		Ausgangsnutzsignal $X_{o}(t)$
		& $X(t)$ 
		& $A_{c}X(t)$
		& \multicolumn{2}{c|}{$A_{c}\mu X(t)$} 
		& \multicolumn{1}{c|} {$k_{p}X(t)$} 
		& {$k_{f}X(t)$}  \\
		\hline
		Leistung $S_{o}$ von $X_{o}(t)$   
		& $S_{X}$
		& $A_{c}^{2} S_{X}$
		& \multicolumn{2}{c|}{$A_{c}^{2}\mu^{2}S_{X}$}
		& \multicolumn{1}{c|} {$k_{p}^{2}S_{X}$}
		& {$k_{f}^{2}S_{X}$} \\
		\hline
		Rauschleistung am Ausgang
		& $\eta B$
		& $2\eta B$
		& \multicolumn{2}{c|}{$2\eta B$}
		& \multicolumn{1}{c|}{$\frac{1}{A_{c}^{2}/2} \eta B$}
		& {$\frac{1}{3}\frac{(2\pi B)^{2}}{A_{c}^{2}/2} \eta B$} \\
		\hline
		SNR am Ausgang $\left(\frac{S}{N}\right)_{o}$
		& $\frac{S_{i}}{\eta B}$
		& $\frac{A_{c}^{2} S_{X}}{2\eta B}$
		& \multicolumn{2}{c|}{$\frac{A_{c}^{2}\mu^{2}S_{X}}{2\eta B}$}
		& \multicolumn{1}{c|}{$\frac{k_{p}^{2}A_{c}^{2}S_{X}}{2\eta B}$}
		& {$\frac{3 D^{2}A_{c}^{2}S_{X}}{2\eta B}$} \\
		\hline
		$\left(\frac{S}{N}\right)_{o}$ ausgedr\"uckt mit  $\gamma = \frac{S_{i}}{\eta B}$
		& $\gamma$
		& $\gamma$
		& \multicolumn{2}{c|}{$\frac{\mu^{2}S_{X}}{1 + \mu^{2}S_{X}}\gamma$}
		& \multicolumn{1}{c|}{$k_{p}^{2}S_{X}\gamma$}
		& {$3 D^{2}S_{X}\gamma$} \\
		\hline 
	\end{longtable}
	Wichtige Anmerkung: Die Formeln der Tabelle gelten f\"ur dimensionslose Signale. Der Zufallsprozess liegt zudem in normierter Form vor, wie aus der Tabelle hervorgeht. Soll die SNR für konkrete physikalisch vorliegende Signale berechnet werden, m\"ussen f\"ur die Amplituden und Leistungen am Eingang des Empf\"angers geeignete Saklierungsfaktoren verwendet werden. Handelt es sich beim Empf\"anger zudem um einen aktiven Schaltungsblock, ist das Signal (sowie der Rauschanteil) am Ausgang des Empf\"angers ebenfalls mit den Parametern des Empf\"angers zu skalieren. \\
	
\renewcommand{\arraystretch}{\arraystretchOriginal}
\end{landscape}
\clearpage
\twocolumn